\documentclass[aspectratio=169]{beamer}

\usepackage{booktabs}
\usepackage{amsmath}

% Load the UConn theme
\usetheme{UConn}

% Optional: enable speaker notes on second screen
% \setbeameroption{show notes on second screen=right}

% Set the footer text and event
\footlinetext{Example Presentation Using UConn Beamer Theme}
\footlineevent{Event'26}

% Presentation Metadata
\title{Your Presentation Title\\Goes Here}
\author{Author Name}
\institute{University of Connecticut}
\date{February 2026}

\begin{document}

%------------------------------------------------------------------------------
% Title Slide (navy background with logo)
%------------------------------------------------------------------------------
\titleslide

%------------------------------------------------------------------------------
% Content Slide: Basic Layout
%------------------------------------------------------------------------------
\begin{frame}{Basic Content Slide}
This is a standard content slide with the UConn theme.

\begin{itemize}
    \item First point with \textbf{bold text}
    \item Second point with \accenttext{accent text} (red + bold)
    \item Third point with \emphtext{emphasis text} (red only)
\end{itemize}

\vspace{0.5em}
\alert{Alert text stands out like this.}
\end{frame}

%------------------------------------------------------------------------------
% Content Slide: Two Columns
%------------------------------------------------------------------------------
\begin{frame}{Two-Column Layout}
\begin{columns}[T]
\column{0.48\textwidth}
\textbf{Left Column}
\begin{itemize}
    \item Supports standard Beamer columns
    \item Works well for text vs.\ figure layouts
\end{itemize}

\column{0.48\textwidth}
\textbf{Right Column}
\begin{enumerate}
    \item Numbered items
    \item Also supported
    \item With UConn styling
\end{enumerate}
\end{columns}
\end{frame}

%------------------------------------------------------------------------------
% Content Slide: Blocks
%------------------------------------------------------------------------------
\begin{frame}{Block Styles}
\begin{block}{Standard Block}
Navy header with light gray body. Used for definitions or key points.
\end{block}

\begin{alertblock}{Alert Block}
Red header with light gray body. Used for warnings or important observations.
\end{alertblock}

\begin{exampleblock}{Example Block}
Lighter navy header. Used for examples or demonstrations.
\end{exampleblock}
\end{frame}

%------------------------------------------------------------------------------
% Content Slide: Challenge/Key Idea Boxes
%------------------------------------------------------------------------------
\begin{frame}{Custom Box Environments}
\begin{columns}[T]
\column{0.48\textwidth}
\begin{challengebox}
\begin{itemize}
    \item Describe the problem
    \item Highlight constraints
\end{itemize}
\end{challengebox}

\column{0.48\textwidth}
\begin{keyideabox}[Solution:]
\begin{itemize}
    \item Describe your approach
    \item Show the key insight
\end{itemize}
\end{keyideabox}
\end{columns}

\vspace{0.5em}
Custom \texttt{challengebox} and \texttt{keyideabox} environments accept an optional title argument.
\end{frame}

%------------------------------------------------------------------------------
% Content Slide: Table
%------------------------------------------------------------------------------
\begin{frame}{Tables}
\begin{table}
\centering
\renewcommand{\arraystretch}{1.3}
\begin{tabular}{lccc}
\toprule
\textbf{Method} & \textbf{Accuracy} & \textbf{Speed} & \textbf{Cost} \\
\midrule
Baseline    & 85.2\% & Slow   & Low \\
Ours        & \accenttext{94.1\%} & Fast   & Low \\
Oracle      & 96.0\% & N/A    & High \\
\bottomrule
\end{tabular}
\end{table}
\end{frame}

%------------------------------------------------------------------------------
% Content Slide: TikZ Diagram
%------------------------------------------------------------------------------
\begin{frame}{TikZ Diagrams}
TikZ libraries are pre-loaded: \texttt{shapes.geometric}, \texttt{arrows.meta}, \texttt{positioning}, \texttt{decorations.pathreplacing}.

\vspace{0.5em}
\centering
\begin{tikzpicture}[scale=0.8, every node/.style={scale=0.8},
    box/.style={draw, rounded corners, minimum width=2.5cm, minimum height=0.6cm, align=center}]
    \node[box, fill=UConnNavy!15] (a) at (0, 0) {Input};
    \node[box, fill=green!30] (b) at (4, 0) {Processing};
    \node[box, fill=UConnAccent!15] (c) at (8, 0) {Output};
    \draw[->, thick, UConnNavy] (a) -- (b);
    \draw[->, thick, UConnNavy] (b) -- (c);
\end{tikzpicture}
\end{frame}

%------------------------------------------------------------------------------
% Content Slide: Animations (Overlays)
%------------------------------------------------------------------------------
\begin{frame}{Step-by-Step Animations}
\textbf{Content appears incrementally:}

\begin{itemize}
    \item This is always visible
    \onslide<2->{\item This appears on click}
    \onslide<3->{\item And this appears on the next click}
\end{itemize}

\onslide<3->{
\begin{alertblock}{Revealed!}
Use \texttt{\textbackslash onslide<n->} to control when content appears.
\end{alertblock}
}
\end{frame}

%------------------------------------------------------------------------------
% Content Slide: Figures
%------------------------------------------------------------------------------
% \begin{frame}{Including Figures}
% \begin{columns}[T]
% \column{0.48\textwidth}
% \begin{figure}
%     \centering
%     \includegraphics[width=\linewidth]{figures/your_figure.pdf}
%     \caption{Left figure}
% \end{figure}
%
% \column{0.48\textwidth}
% \begin{figure}
%     \centering
%     \includegraphics[width=\linewidth]{figures/another_figure.pdf}
%     \caption{Right figure}
% \end{figure}
% \end{columns}
% \end{frame}

%------------------------------------------------------------------------------
% Thank You Slide (navy background with logo)
%------------------------------------------------------------------------------
\thankyouslide[Author Name \quad\texttt{author@uconn.edu}]

\end{document}
